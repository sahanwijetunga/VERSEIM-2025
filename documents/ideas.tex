% Created 2025-04-19 Sat 21:50
% Intended LaTeX compiler: pdflatex
\documentclass[11pt]{article}
\usepackage[utf8]{inputenc}
\usepackage[T1]{fontenc}
\usepackage{graphicx}
\usepackage{longtable}
\usepackage{wrapfig}
\usepackage{rotating}
\usepackage[normalem]{ulem}
\usepackage{amsmath}
\usepackage{amssymb}
\usepackage{capt-of}
\usepackage{hyperref}

%%--------------------------------------------------------------------------------

\usepackage[svgnames]{xcolor}
\usepackage{mathrsfs}
\usepackage{tikz-cd}
\usepackage{svg}
\usepackage[top=25mm,bottom=25mm,left=25mm,right=30mm]{geometry}


\usepackage{amsthm}
\usepackage{thmtools}
\usepackage{cleveref}
\usepackage{stmaryrd}

%%\usepackage[outputdir=build]{minted}
\usepackage{minted}
\usemintedstyle{tango}
\usepackage[svgnames]{xcolor}
\setminted[bash]{bgcolor=NavajoWhite}
\setminted[python]{bgcolor=Lavender}
\setminted[sage]{bgcolor=Lavender}

\usepackage{titlesec}
\newcommand{\sectionbreak}{\clearpage}


\renewcommand{\descriptionlabel}[1]{\hspace{\labelsep}#1}


\usepackage[cal=boondox]{mathalfa}


\newenvironment{referee}{\color{red}}{\color{black}}

\numberwithin{equation}{section}

\declaretheorem[numberwithin=subsection,Refname={Theorem,Theorems}]{theorem}
\declaretheorem[sibling=theorem,name=Proposition,Refname={Proposition,Propositions}]{proposition}
\declaretheorem[sibling=theorem,name=Corollary,Refname={Corollary,Corollaries}]{corollary}
\declaretheorem[sibling=theorem,name=Lemma,Refname={Lemma,Lemmas}]{lemma}
\declaretheorem[sibling=theorem,name=Remark,style=remark,Refname={Remark,Remarks}]{remark}
\declaretheorem[sibling=theorem,name=Problem,style=remark,Refname={Problem,Problems}]{problem}
\declaretheorem[sibling=theorem,name=Example,style=remark,Refname={Example,Examples}]{ex}
\declaretheorem[sibling=theorem,name=Definition,style=remark,Refname={Definition,Definitions}]{definition}

\crefname{equation}{}{}

%%--------------------------------------------------------------------------------

\newenvironment{solution}
{\par \color{red}\hrulefill \newline \noindent \textbf{Solution:} \vspace{2mm}}
{\vspace{2mm} \color{black}}


\newcommand{\totdeg}{\operatorname{totdeg}}
\newcommand{\content}{\operatorname{content}}

\newcommand{\Mat}{\operatorname{Mat}}

\newcommand{\Aut}{\operatorname{Aut}}
\newcommand{\Gal}{\operatorname{Gal}}

\newcommand{\A}{\mathscr{A}}
\newcommand{\B}{\mathscr{B}}
\newcommand{\FF}{\mathscr{F}}
\newcommand{\LF}{\mathcal{LF}}

\newcommand{\HH}{\mathcal{H}}
\newcommand{\X}{\mathscr{X}}

\newcommand{\ff}{\mathfrak{f}}
\newcommand{\pp}{\mathfrak{p}}

\newcommand{\Z}{\mathbf{Z}}
\newcommand{\Q}{\mathbf{Q}}
\newcommand{\R}{\mathbf{R}}
\newcommand{\C}{\mathbf{C}}
\newcommand{\F}{\mathbf{F}}

\newcommand{\Poly}{\mathcal{P}}
%%--------------------------------------------------------------------------------
\author{George McNinch}
\date{2025-04-19 21:50:08 EDT (george@valhalla)}
\title{Some formalization ideas for VERSEIM-2025}
\hypersetup{
 pdfauthor={George McNinch},
 pdftitle={Some formalization ideas for VERSEIM-2025},
 pdfkeywords={formalization, algebra, finite-algebra},
 pdfsubject={},
 pdfcreator={Emacs 30.0.92 (Org mode 9.7.11)}, 
 pdflang={English}}
\usepackage{calc}
\newlength{\cslhangindent}
\setlength{\cslhangindent}{1.5em}
\newlength{\csllabelsep}
\setlength{\csllabelsep}{0.6em}
\newlength{\csllabelwidth}
\setlength{\csllabelwidth}{0.45em * 0}
\newenvironment{cslbibliography}[2] % 1st arg. is hanging-indent, 2nd entry spacing.
 {% By default, paragraphs are not indented.
  \setlength{\parindent}{0pt}
  % Hanging indent is turned on when first argument is 1.
  \ifodd #1
  \let\oldpar\par
  \def\par{\hangindent=\cslhangindent\oldpar}
  \fi
  % Set entry spacing based on the second argument.
  \setlength{\parskip}{\parskip +  #2\baselineskip}
 }%
 {}
\newcommand{\cslblock}[1]{#1\hfill\break}
\newcommand{\cslleftmargin}[1]{\parbox[t]{\csllabelsep + \csllabelwidth}{#1}}
\newcommand{\cslrightinline}[1]
  {\parbox[t]{\linewidth - \csllabelsep - \csllabelwidth}{#1}\break}
\newcommand{\cslindent}[1]{\hspace{\cslhangindent}#1}
\newcommand{\cslbibitem}[2]
  {\leavevmode\vadjust pre{\hypertarget{citeproc_bib_item_#1}{}}#2}
\makeatletter
\newcommand{\cslcitation}[2]
 {\protect\hyper@linkstart{cite}{citeproc_bib_item_#1}#2\hyper@linkend}
\makeatother\begin{document}

\maketitle
\setcounter{tocdepth}{2}
\tableofcontents

\section{Warm-up problems}
\label{sec:warm-up-problems-}
\subsection{linear algebra results}
\label{sec:linear-algebra-results}
\begin{itemize}
\item formalize correspondence between linear transformations \(V \to W\)
and matrices (where \(V\) and \(W\) are finite dimensional vector spaces
over some field).

\item e.g. formluate and prove statements about eigenvectors and
eigenvalues of a linear endomorphism of a finite dimensional vector
space V.

Probably the ultimate target would be the Cayley-Hamilton theorem.
\end{itemize}
\subsection{finite group theory}
\label{sec:finite-group-theory}
\begin{itemize}
\item prove that a finite p-group has a non-trivial center (and hence that
a finite p-group is solvable)

\item for a finite p-group G and a field k of char p>0, prove that for any
finite dimensional k-vector space V and any homomorphism \(\rho:G
  \to GL(V)\) that \(G\) fixes a non-zero vector in \(V\).
\end{itemize}
\subsection{commutative rings}
\label{sec:commutative-rings}
\begin{itemize}
\item prove the Gauss Lemma and Eisenstein's criteria
\href{https://leanprover-community.github.io/mathlib4\_docs/Mathlib/RingTheory/Polynomial/Content.html\#Polynomial.content}{Gauss Lemma already has a proof in mathlib}
and so does \href{https://leanprover-community.github.io/mathlib4\_docs/Mathlib/RingTheory/Polynomial/Eisenstein/Basic.html\#Polynomial.IsEisensteinAt.irreducible}{Eisenstein's critera}
\end{itemize}
\subsection{Graph theory}
\label{sec:graph-theory}
\begin{itemize}
\item mathlib has a proof of \href{https://leanprover-community.github.io/mathlib4\_docs/Mathlib/Combinatorics/Hall/Finite.html}{Hall's Marriage Theorem}
\item prove elementary fact: sum of degrees of vertices is twice the
number of edges.
\end{itemize}
\section{Formalization ideas}
\label{sec:formalization-ideas}
\subsection{\href{https://en.wikipedia.org/wiki/Projective\_space}{Projective spaces} and \href{https://en.wikipedia.org/wiki/Grassmannian}{grassmannians}}
\label{sec:projective-spaces-and-grassmannians}
\begin{itemize}
\item mathlib has a formalization of \href{https://leanprover-community.github.io/mathlib4\_docs/Mathlib/LinearAlgebra/Projectivization/Basic.html\#Projectivization}{projective spaces} can we imitate this
formalization to the Grassmannian? What results should be proved
about it?

Say something about Plücker embedding?

Is there already a formalization of the \emph{exterior powers} of a vector
space? Surely\ldots{}
\end{itemize}
\subsection{forms over a finite field}
\label{sec:forms-over-a-finite-field}
Defined on a finite dimensional vector space \(V\) over a finite field

\begin{itemize}
\item reflexive forms
\item quadratic forms / symmetric forms (char 2 and \(p>2\))
\item alternating forms
\item Hermitian forms

\item Can you give a formal proof of the theorem describing the number of
points of a quadric (i.e. the zero set in \(\mathbb{P}(V)\) of a
non-degenerate quadratic form \(q\) on the vector space \(V\) over a
finite field.
\end{itemize}
\subsection{Polar spaces}
\label{sec:polar-spaces}
\begin{itemize}
\item such a space is a ``point-line geometry''. Formalize the notion of
point-line geometry.

\item polar spaces arise from reflexive form on a vector space on a finite
field.
\end{itemize}
\subsection{Fourier transforms for functions on vector spaces over a finite field}
\label{sec:fourier-transforms-for-functions-on-vector-spaces-over-a-finite-field}
\subsection{Algebraic combinatorics}
\label{sec:algebraic-combinatorics}
\begin{itemize}
\item formalize proof of some results from book of R. Stanley.

e.g. Theorem 1.1 which gives a condition for a formal power series
\(f \in k\llbracket t \rrbracket\) to be a \emph{rational function.} Try
to formalize this proof.

or more generally, all the ``tool-results'' from the first section of
Stanley's book.
\end{itemize}
\subsection{error-correcting codes}
\label{sec:codes}
\begin{itemize}
\item formalize some basic results about codes - see Simeon Ball's book.
\end{itemize}
\subsection{quaternion algebras}
\label{sec:quaternion-algebras}
\begin{itemize}
\item show they are simple

\item describe in the form \((a,b)\) or (in char. 2) \((a,b]\).

\item give criteria on \(a,b\) for when the algebra is division.

\item formalize proof of result from P. Gilles book about quadratic forms
\& quaternion algebras.
\end{itemize}
\end{document}
